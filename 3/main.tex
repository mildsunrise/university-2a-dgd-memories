\chapter{Pràctica 3: Disseny d'un joc}

\section{Especificació}

% TODO

Es farà servir una placa DE2 on es carregarà el resultat. Es farà servir el Quartus 9.1 SP2 Web Edition per al desenvolupament.

\section{Implementació}

% TODO

\section{Blocs importats}

En aquesta secció es llisten els blocs que s'han importat de pràctiques
anteriors i es fan servir sense modificar-los (ni modificar les seves
dependències). Per aquest motiu, no s'inclou la implementació ni la
simulació d'aquests.

Un bloc importat només apareix si es fa servir directament en algun bloc
no importat. Per tant, les seves dependències no apareixen necessàriament
en aquesta memòria.

\inputblock{imported/f_div}
\inputblock{imported/keytest}
\inputblock{imported/BCD7seg}
\inputblock{imported/hex_disps}

\section{Blocs del joc}
\inputblock{joc/keygroup}
\inputblock{joc/comptador}
\inputblock{joc/registres}
\inputblock{joc/comparador}
\inputblock{joc/control}
\inputblock{joc}

\section{Blocs d'adaptació a la placa}
\inputblock{leds}
\inputblock{jocDE2}

\section{Valoració general}

L'únic problema rellevant era que el comptador feia l'overflow en les unitats
però no en les desenes (passava de \texttt{0x99} a \texttt{0xA0}), i de vegades
el joc quedava impossible de resoldre. Solucionat això, tot va funcionar correctament.
