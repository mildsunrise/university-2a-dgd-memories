\chapter{Pràctica 2: Multiplicador BCD seqüencial}

\section{Especificació}

Es demana dissenyar un multiplicador de dues xifres BCD a la placa FPGA. L'usuari introduirà manualment els operands a través d'un teclat. Els operands d'entrada es mostraran a través dos displays set-segments de la placa i la sortida es mostrarà en uns altres dos. El funcionament ha de ser el següent:

Quan es prem la tecla \texttt{*}, s'apaguen els displays on es mostra el resultat i el sistema espera la introducció de les xifres. A partir d'aquí el sistema emmagatzema les dues darreres xifres que anem introduint al teclat, i les mostra als displays corresponents. Aquestes xifres constituiràn els factors, el producte dels quals es mostrarà als displays de sortida quan polsem la tecla \texttt{\#}.

Addicionalment, mentre sigui possible introduir operands al teclat, es mantidran encesos quatre LEDs verds. Altrament, s'encendran cuatre LEDs vermells.

Exemple: si es prem \texttt{*}, \texttt{8}, \texttt{3}, \texttt{\#} els displays mostraran:

\begin{center}
\sevenseg{ }\sevenseg{3}\sevenseg{ }\sevenseg{8}\sevenseg{ }\sevenseg{ }\sevenseg{2}\sevenseg{4}
\end{center}

Es farà servir una placa DE2 on es carregarà el resultat. Es farà servir el Quartus 9.1 SP2 Web Edition per al desenvolupament.

\section{Implementació}

Per tractar la entrada haurem de realitzar el bloc \textsf{keygroup}, que s'encarregarà d'interpretar el tipus de tecla que es prem. Aquest es conectarà a \textsf{control}, que es un bloc donat que coordinarà els diversos blocs que formen el conjunt que hem anomenat \textsf{ppal}, per tal que funcioni tal i com s'ha descrit anteriorment. Les sortides de \textsf{control} aniràn a \textsf{leds}, \textsf{sel} i \textsf{regs}.

Per altra banda el bloc \textsf{regs} anirà guardant en dos registres les dues últimes xifres que l'usuari vagi introduïnt (quan $intro$ estigui activat després de premer \texttt{*}). Les dues sortides d'aquest bloc aniràn al multiplicador \textsf{AperB}.

Per fer el multiplicador reutilitzarem alguns blocs de la pràctica anterior, i els hem d'adaptar per a aquesta implementació. Ara l'entrada es troba en BCD d'una xifra, per tant el rang de les entrades amb els que operarem será $\left[0,9\right]$, mentre que el rang de la sortida serà $\left[0,81\right]$. En l'esquemàtic del multiplicador haurem d'emplenar amb zeros en comptes d'estendre el signe, i afegir els valors que falten al conversor de Ca2 a BCD que es troba a la sortida del multiplicador. 

Per últim s'hauran d'implementar el bloc \textsf{sel}, que mostrarà la sortida només quan $show$ estigui activat.

Tots els blocs esmentats anteriorment pertanyen a un grup que hem anomenat \textsf{ppal}. Perquè aquest bloc pugui funcionar, li haurem d'afegir un clock, i un bloc que mostri la sortida a la placa, així com un altre que s'encarregui dels LEDs d'aquesta.

\section{Blocs importats}

En aquesta secció es llisten els blocs que s'han importat de pràctiques
anteriors i es fan servir sense modificar-los (ni modificar les seves
dependències). Per aquest motiu, no s'inclou la implementació ni la
simulació d'aquests.

Un bloc importat només apareix si es fa servir directament en algun bloc
no importat. Per tant, les seves dependències no apareixen necessàriament
en aquesta memòria.

\inputblock{imported/MULT_8x8}
\inputblock{imported/BCD7seg}

\section{Blocs combinacionals}
\inputblock{ppal/CA2_BCD_8B}
\inputblock{ppal/keygroup}
\inputblock{ppal/AperB}
\inputblock{ppal/sel}

\section{Blocs seqüencials}
\inputblock{ppal/control}
\inputblock{ppal/regs}
\inputblock{ppal}

\section{Blocs d'adaptació a la placa}
\inputblock{f_div}
\inputblock{keytest}
\inputblock{leds}
\inputblock{hex_disps}
\inputblock{calc}

\section{Valoració general}

Cap problema rellevant, excepte una falta a la descripció de \textsf{keygroup} que
va detectar-se en una última revisió abans de pujar-lo a la placa.
