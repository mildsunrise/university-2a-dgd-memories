\chapter{Millora: Formateig natural de sortida}

\section{Especificació}

Continunant amb el resultat obtingut en la pràctica 2, i de forma similar a la
millora implementada en la pràctica 1B, ens proposem obtenir una
representació més natural (propera al que s'escriu en la realitat) del nombre de
sortida. Bàsicament ens agradaria eliminar el zero redundant, mantenint el signe
(si n'hi ha) el més a la dreta possible.

Exemples de valors, amb la seva representació actual i la que ens agradaria
aconseguir:

\begin{center}
\begin{tabular}{ccc} \toprule
Valor & Abans & Després \\ \midrule
-35 & \sevenseg{-}\sevenseg{3}\sevenseg{5}
    & \sevenseg{-}\sevenseg{3}\sevenseg{5} \\ \midrule
-7  & \sevenseg{-}\sevenseg{0}\sevenseg{7}
    & \sevenseg{ }\sevenseg{-}\sevenseg{7} \\ \midrule
-1  & \sevenseg{-}\sevenseg{0}\sevenseg{1}
    & \sevenseg{ }\sevenseg{-}\sevenseg{1} \\ \midrule
0   & \sevenseg{ }\sevenseg{0}\sevenseg{0}
    & \sevenseg{ }\sevenseg{ }\sevenseg{0} \\ \midrule
3   & \sevenseg{ }\sevenseg{0}\sevenseg{3}
    & \sevenseg{ }\sevenseg{ }\sevenseg{3} \\ \midrule
10  & \sevenseg{ }\sevenseg{1}\sevenseg{0}
    & \sevenseg{ }\sevenseg{1}\sevenseg{0} \\ \midrule
28  & \sevenseg{ }\sevenseg{2}\sevenseg{8}
    & \sevenseg{ }\sevenseg{2}\sevenseg{8} \\ \bottomrule
\end{tabular}
\end{center}

\section{Implementació}

Aquesta millora es pot implementar simplement modificant \textsf{hex\_disps}, el
bloc responsable del formateig de la sortida. Si $|x| > 9$, el nombre és de dos
xifres i per tant la representació es la mateixa que l'actual. En cas contrari,
la xifra de les desenes s'ha de remplaçar pel signe, i el display on anava el signe
s'ha d'apagar completament.

  \cclearpage
\section{Blocs d'adaptació a la placa}
\inputblock{hex_disps}

\section{Valoració general}

Cap problema rellevant. La millora era molt similar a la de l'anterior pràctica,
i tot i que vam haver d'escriure-la de nou al VHDL de \textsf{hex\_disps}, va
funcionar sense problemes un cop compilava.
