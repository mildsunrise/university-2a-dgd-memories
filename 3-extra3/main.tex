\chapter{Extra 3: Funcionalitat de trampa}

\section{Especificació}

Es demana modificar el sistema per incloure un «mode trampa» que ensenya temporalment el nombre que s'ha d'endevinar a través dels displays.

En mig d'una partida, es pot premer una de les lletres del teclat (\texttt{A}, \texttt{B}, \texttt{C} o \texttt{D}) i els displays visualitzaran
la solució. El nombre haurà de desaparèixer quan es tornir a prèmer una d'aquestes tecles, tornant a la funcionalitat de joc.

\paragraph{Exemple de funcionament.}

A continuació es mostra una partida d'exemple, on es veu l'estat dels dos displays
set-segments i els LEDs verds de la placa. S'inicia una partida, es prova alguns nombres,
es visualitza la trampa, es torna al funcionament normal i s'encerta la solució.

\begin{multicols}{2}

$\rightarrow$ Reset.

\runrow{0}{0}{{X}{X}{X}{X}{X}{X}{X}{X}}

\keypress{\#}

\runrow{0}{0}{{ }{ }{ }{ }{ }{ }{ }{ }}

\keypress{3}

\runrow{0}{3}{{ }{ }{ }{ }{ }{ }{ }{ }}

\keypress{6}

\runrow{3}{6}{{ }{ }{ }{ }{ }{ }{ }{ }}

\keypress{*}

\runrow{3}{6}{{X}{X}{X}{X}{ }{ }{ }{ }}

\keypress{5}

\runrow{6}{5}{{ }{ }{ }{ }{ }{ }{ }{ }}

\keypress{1}

\runrow{5}{1}{{ }{ }{ }{ }{ }{ }{ }{ }}

\keypress{*}

\runrow{5}{1}{{ }{ }{ }{ }{X}{X}{X}{X}}

\keypress{B}

\runrow{4}{4}{{ }{ }{ }{ }{ }{ }{ }{ }}

\keypress{B}

\runrow{5}{1}{{ }{ }{ }{ }{ }{ }{ }{ }}

\keypress{4}

\runrow{1}{4}{{ }{ }{ }{ }{ }{ }{ }{ }}

\keypress{4}

\runrow{4}{4}{{ }{ }{ }{ }{ }{ }{ }{ }}

\keypress{*}

\runrow{4}{4}{{ }{ }{X}{X}{X}{X}{ }{ }}

\end{multicols}


\section{Implementació}

D'entrada, haurem de modificar el bloc \textsf{keygroup} perquè sigui capaç de detectar el nou tipus de tecla premuda.
S'inclourà una sortida nova $let$ que anirà també a \textsf{control} amb les altres.

Llavors hem d'incloure un nou estat, que anomenarem \emph{cheating}, en la màquina de control.
Aquesta tindrà una nova sortida ($show$) que indicarà si la trampa està activada.

Finalment, només cal fer que $show$ seleccioni la sortida a visualitzar entre $num$ i $numx$.

\section{Blocs del joc}
\inputblock{joc/keygroup}
\inputblock{joc/control}
\inputblock{joc}

\section{Valoració general}

No ens vàrem trobar cap problema rellevant.
