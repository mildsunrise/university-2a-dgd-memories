\chapter{Pràctica 1B: Multiplicador CA2}

\section{Especificació}

Es demana construir un multiplicador de 2 paraules de 4~bits en complement~a~2 per
a la placa FPGA. L'usuari introduïrà manualment les dues paraules mitjançant
4~switches per a cada una, i els displays de set segments de la placa mostraran
les dues entrades, i el resultat de la multiplicació (en format signe - modul
decimal habitual).

S'han de prendre les precaucions a l'hora de calcular el producte, i tants displays
set-segments com sigui necessari per a que, per a qualsevol parella d'entrades
possible, la sortida no estigui fora de rang i es visualitzi correctament.

Es farà servir una placa DE2 on es carregarà el resultat. Es farà servir el
Quartus 9.1 SP2 Web Edition per al desenvolupament.

\section{Implementació}

Donat que les entrades son Ca2 de quatre bits, el seu rang és $\left[-8,7\right]$
i per tant el resultat de la multiplicació estarà dins el rang $\left[7 \cdot -8, -8 \cdot -8\right] = \left[-56, 64\right]$. Per poder representar nombres de fins a 64 ens cal operar amb Ca2 a 8 bits, què té rang $\left[-128, 127\right]$.

Així doncs, ens cal un convertidor de codi, de Ca2 amb 4~bits a Ca2 amb 8~bits.
Un cop fet això, podem dur a terme la multiplicació amb un multiplicador de 8~bits
i obtindriem el resultat, també en Ca2 amb 8~bits.

Ara ens cal representar el signe i el mòdul de les entrades i la sortida als 
displays set-segments de la placa. Delegarem aquesta
tasca a blocs específics, un per a Ca2 amb 4~bits i un altre per a Ca2 amb 8~bits.
En el primer cas, només ens cal reservar 2 displays: un pel signe, l'altre per la
única xifra del nombre. En el segon cas però, necessitem dos xifres i per tant
3~displays reservats de sortida.

En tots dos casos necessitem començar extraient el signe i el valor absolut del nombre passat. Llavors convertim el signe al patró apropiat per al display que
l'ha de representar. Només queda representar el mòdul. En el primer cas és senzill;
el mòdul ja és una xifra BCD, només cal convertir-la mitjançant el bloc
\textsf{BCD7seg}.

En el segon cas però, cal primer convertir el mòdul a BCD
mitjançant el bloc (també proporcionat) \textsf{CA2\_BCD\_8B}, la sortida son dues
xifres que hem de separar i portar, cadascuna, a una instància de \textsf{BCD7seg}
i al seu display.

A continuació es descriuen els blocs que s'han emprat, començant pels menys
simples i acabant amb el bloc de més alt nivell jeràrquic, que es carrega
directament a la placa.

  \cclearpage
\section{Blocs d'aritmètica}
\inputblock{SUM_1B}
  \cclearpage
\inputblock{SUM_4B}
  \cclearpage
\inputblock{SUM_8B}
  \cclearpage
\inputblock{MULT_8x1}
  \cclearpage
\inputblock{MULT_ACC}
  \cclearpage
\inputblock{MULT_8x8}

  \cclearpage
\section{Blocs de codificació}
\inputblock{CA2_4A8}
  \cclearpage
\inputblock{CA2_BCD_4B}
  \cclearpage
\inputblock{CA2_BCD_8B}

  \cclearpage
\section{Blocs de presentació}
\inputblock{BCD7seg}
  \cclearpage
\inputblock{CA2_SIG_SS}
  \cclearpage
\inputblock{CA2_SS_4B}
  \cclearpage
\inputblock{CA2_SS_8B}
  \cclearpage
\inputblock{MULT_CA2_BCD}

\section{Valoració general}

Per fer la pràctica, ens donaven fet el bloc \textsf{CA2\_BCD\_8B}, però conforme aquesta anava avançant ens va semblar més còmode treballar amb un codificador \textsf{CA2\_BCD\_4B}, ja que era més directe en certs cassos. Llavors, una vegada creat el \textsf{CA2\_BCD\_4B}, ja haviem fet alguns esquemàtics que feien servir el de 8~bits i simplement els vàrem adaptar per treballar amb el de 4~bits.

