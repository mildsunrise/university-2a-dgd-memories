\subsection{\label{sub:\projectname-CA2_4A8} \textsf{CA2\_4A8}}

\paragraph{Símbol}
\begin{center} \bsfsymbol{CA2_4A8} \end{center}

\paragraph{Entrades i sortides}

\begin{where}
\item[\nodenamerange{a}{3}{0}] Entrada (Ca2 4~bits)
\item[\nodenamerange{z}{7}{0}] Sortida (Ca2 8~bits)
\end{where}

\paragraph{Funció}

Canviador de mida del codi, de Ca2 4~bits a Ca2 8~bits.

Expressa el valor de l'entrada (complement a 2 amb 4~bits) en el codi
de sortida (complement a 2 amb 8~bits).

\paragraph{Inespecificacions}

Cap.

\paragraph{Implementació}

\begin{figure}[b]
  \begin{center}
    \adjustbox{max width=\textwidth, max height=\textheight}{
      \bdfschematic{CA2_4A8}
    }
  \end{center}
  \caption{\label{fig:\projectname-CA2_4A8} Esquemàtic per al bloc \textsf{CA2\_4A8}}
\end{figure}

L'esquemàtic del bloc es pot veure a la figura~\ref{fig:\projectname-CA2_4A8} (pàgina~\pageref{fig:\projectname-CA2_4A8}).

Només cal fer una extensió de signe; l'entrada es copia directament als 4~bits
de menys pes de la sortida, i els 4~bits restants s'emplenen amb el signe (o sigui,
$a_3$).

\vspace{1cm}
