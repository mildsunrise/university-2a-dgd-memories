\subsection{\label{sub:\projectname-CA2_SS_4B} \textsf{CA2\_SS\_4B}}

\paragraph{Símbol}
\begin{center} \bsfsymbol{CA2_SS_4B} \end{center}

\paragraph{Entrades i sortides}

\begin{where}
\item[\nodenamerange{x}{3}{0}] Entrada (Ca2)
\item[\nodenamerange{hex0}{6}{0}] Primer display en ordre lèxic (set-segments actiu baix, signe)
\item[\nodenamerange{hex1}{6}{0}] Segon display en ordre lèxic (set-segments actiu baix, unitats)
\end{where}

\paragraph{Funció}

Formatejador de Ca2 4~bits en set-segments.

Donat un valor en complement a 2 de 4~bits, el representa en format signe --
mòdul decimal habitual en 2 displays set-segments $hex0$ i $hex1$.

Nota: Els displays estan numerats en ordre lèxic ($hex0$ és el de l'esquerra).

\paragraph{Inespecificacions}

Cap.

\paragraph{Implementació}

\begin{figure}[b]
  \begin{center}
    \adjustbox{max width=\textwidth, max height=\textheight}{
      \bdfschematic{CA2_SS_4B}
    }
  \end{center}
  \caption{\label{fig:\projectname-CA2_SS_4B} Esquemàtic per al bloc \textsf{CA2\_SS\_4B}}
\end{figure}

L'esquemàtic del bloc es pot veure a la figura~\ref{fig:\projectname-CA2_SS_4B} (pàgina~\pageref{fig:\projectname-CA2_SS_4B}).

En primer lloc, el signe $x_3$ es fa entrar al bloc \textsf{CA2\_SIG\_SS} i la
representació resultant es retorna a $hex0$. Llavors, s'utilitza el bloc
\textsf{CA2\_BCD\_4B} per a obtenir el mòdul de $x$. Aquest mòdul ja és una xifra
BCD vàlida, que es converteix a set-segments amb \textsf{BCD7seg} i es retorna a
$hex1$.

\vspace{1cm}
