\subsection{\label{sub:\projectname-MULT_CA2_BCD} \textsf{MULT\_CA2\_BCD}}

\paragraph{Entrades i sortides}

\begin{where}
\item[\nodenamerange{SW}{17}{14}] Switches on s'introdueix el primer factor en Ca2
\item[\nodenamerange{SW}{13}{10}] Switches on s'introdueix el segon factor en Ca2
\item[\nodenamerange{HEX7}{6}{0}] Display set-segments 7
\item[\nodenamerange{HEX6}{6}{0}] Display set-segments 6
\item[\nodenamerange{HEX5}{6}{0}] Display set-segments 5
\item[\nodenamerange{HEX4}{6}{0}] Display set-segments 4
\item[\nodenamerange{HEX3}{6}{0}] Display set-segments 3
\item[\nodenamerange{HEX2}{6}{0}] Display set-segments 2
\item[\nodenamerange{HEX1}{6}{0}] Display set-segments 1
\item[\nodenamerange{HEX0}{6}{0}] Display set-segments 0
\end{where}

\paragraph{Funció}

Disseny a carregar a la placa FPGA.

Visualitza els dos factors en format signe -- modul habitual, en els displays
7--6 i 5--4 respectivament. Multiplica els dos factors i visualitza el resultat
amb la mateixa representació, en els displays 2--0. El display restant està
sempre apagat.

\paragraph{Inespecificacions}

Cap.

\paragraph{Implementació}

\begin{figure}[b]
  \begin{center}
    \adjustbox{max width=\textwidth, max height=\textheight}{
      \bdfschematic{MULT_CA2_BCD}
    }
  \end{center}
  \caption{\label{fig:\projectname-MULT_CA2_BCD} Esquemàtic per al bloc \textsf{MULT\_CA2\_BCD}}
\end{figure}

L'esquemàtic del bloc es pot veure a la figura~\ref{fig:\projectname-MULT_CA2_BCD} (pàgina~\pageref{fig:\projectname-MULT_CA2_BCD}).

Ambdues entrades es porten cap a blocs \textsf{CA2\_4A8} per a convertir-les a
Ca2 abans d'operar. Llavors, es fan entrar a un bloc \textsf{MULT\_8x8}.

Finalment, les entrades es porten també a blocs \textsf{CA2\_SS\_4B} i llavors
als displays corresponents. La sortida es porta a un bloc \textsf{CA2\_SS\_8B}
i als displays corresponents.

\vspace{1cm}
