\subsection{\label{sub:\projectname-registres} \textsf{registres}}

\paragraph{Símbol}

\begin{center} \bsfsymbol{registres} \end{center}

\paragraph{Entrades i sortides}

\begin{where}
\item[\nodenamebit{eshft}] Habilitació d'emmagatzemament
\item[\nodenamerange{keycode}{3}{0}] Xifra que s'està introduint (BCD)
\item[\nodenamerange{num}{7}{0}] Nombre enmagatzemat (BCD, dos xifres)
\item[\nodenamebit{clk}] Rellotge, flanc de pujada
\item[\nodenamebit{nrst}] Reset asíncron, actiu baix
\end{where}

\paragraph{Funció}

\emph{Shift register} per a dues xifres BCD.

Emmagatzema un nombre BCD de dues xifres, memoritzant les dues últimes
xifres carregades. Quan s'habilita la càrrega ($eshft = 1$), la xifra
BCD present a $keycode$ esdevé la xifra de menys pes del nombre
emmagatzemat; la xifra anterior esdevé la de més pes, i aquesta es descarta.

\paragraph{Inespecificacions}


La sortida no pertanyirà al seu codi si es carrega una xifra no BCD.


\paragraph{Notes}

La simulació d'aquest bloc no s'inclou ja que és idèntica a la anterior (llevat espuris,
el comportament no ha canviat).

\paragraph{Implementació}

\vhdlisting{registres}



Implementació senzilla, s'utilitzen dues senyals intermèdies $a$ i $b$
per desar les dues xifres. En el \mintinline{vhdl}|process| s'assignen condicionalment a $a$
i $keycode$ respectivament, i fora d'aquest es concatenen per formar la sortida.

\vspace{1cm}
